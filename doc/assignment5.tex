\documentclass{article}

\usepackage{graphicx}
\usepackage{fullpage}
\usepackage{multicol}
\usepackage{hyperref}

\setlength{\textwidth}{6.5in}
\setlength{\textheight}{8.6in}
\setlength{\topmargin}{0.0in}
\setlength{\oddsidemargin}{0in}
\setlength{\evensidemargin}{0in}

\begin{document}

\noindent
\framebox{\framebox{\vbox{
\begin{tabular}{l}
{\bf{\Large Operating Systems, Assignment 5}} \\
\end{tabular}
}}}

\vspace{0.1in}

This assignment includes two programming problems, one in C
and one in Java.  As always, be sure your code compiles and
runs on the EOS Linux machines before you submit.  Also, remember that
your work needs to be done individually and that your programs need to
be commented and consistently indented.  You \textbf{must} document
your sources and, if you use sources, you must still write at least
half your code yourself from scratch.

\begin{enumerate}
\item (16 pts) Using your textbook, please give a short definition of
  the following terms, in your own words.  Write up all your answers
  into a plain text file called \texttt{terms.txt} and submit it using
  WolfWare classic when you're done.

\begin{enumerate}
\item computation migration

%

\item dynamic routing

% 

\item limited copy (for access rights)

% 

\item breach of availability

% 

\end{enumerate}

\item (5 pts) Arc account activation.

  On our next homework assignment, you'll need to use a
  high-performance computing cluster called arc.  Before you can do
  this, you'll need to follow steps to activate your account on arc
  and set up a key pair so you can log in.  If you don't complete arc
  activation on this assignment, there's a problem you won't be able
  to do on the next assignment.  To help make sure everyone
  is ready, I have a low-point-value problem on this assignment,
  requiring you to activate your account on arc and log in on the head
  node of the arc cluster.

  About a week before this assignment is due, you should receive an
  email from the system administrator on arc.  It will contain a link
  to more information about the arc cluster (most of which won't apply
  to you) and a link to enter a public key to activate your account.
  First, you'll need to generate an RSA key pair.

  If you've never generated a ssh keypair before, it's easy (if you've
  done it before, keeping multiple keys around takes a little more
  work, but you probably know what you're doing if you've done this
  before).  To generate your key pair, log in on one of the EOS linux systems,
  and enter the command:

\begin{verbatim}
ssh-keygen -t rsa -b 4096
\end{verbatim}

  This command will create a 4096-bit RSA key pair, and store it in a
  couple of files in the .ssh directory under your home directory.

  Have a look at the public key of this pair:

\begin{verbatim}
cat .ssh/id_rsa.pub
\end{verbatim}

  Click the link in your email for activating your account.  Your web
  browser will probably complain that it's seeing a self-signed
  certificate; that's OK.  Once you get to the destination page, paste
  the contents of your \verb+.ssh/id_rsa.pub+ file into the text
  box (the entire output of the cat command above).  Press the button
  at the bottom of the form and your account should be activated.

  Then, from one of the university Linux machines, you should be able
  to ssh into the head node on the arc system:

\begin{verbatim}
  ssh arc.csc.ncsu.edu
\end{verbatim}

  If this works, you'll get a shell prompt on the arc system (you need
  to see this shell prompt on arc to get credit for this problem).  If
  this works, you should be in good shape for the next assignment.
  You can log out of the arc system by entering \texttt{exit}.
  
\item (40 pts) We're going to implement a multi-threaded Unix
  client/server program in C using TCP/IP sockets for communication.
  The program will let clients lock and unlock various items.  If a
  client tries to lock an item that's already locked, the server will
  make that client wait until the requested item is unlocked.  I'm
  providing you a skeleton of the server to help get you started,
  \texttt{lockServer.c}.  You'll complete the implementation of this
  server, adding support for multi-threading and synchronization, and
  building some
  representation for the set of items that are currently locked.  You
  won't need to write a client; for
  that, we're going to use a general-purpose program for network
  communication, \texttt{nc}.

  Once started, your server will run perpetually, accepting
  connections from clients until the user kills it with ctrl-C.  The
  program we're using as a client, \texttt{nc}, will take the server's
  hostname and port number as command-line arguments.  After
  connecting to the server, nc will just echo to the screen any text
  sent by the server and send any text the user types to the server.

  While a client is connected, the server will repeatedly prompt for a
  command using the prompt ``\verb$cmd> $''.  In the starter code, the server just
  echoes each command back to the client, but you're going to modify
  it to support the following commands:

\begin{itemize}
\item \texttt{lock} \textit{name}\\
  Wait until it's possible to lock the item with the given name (i.e.,
  until it's not locked), then add this item to the set of locked items.
  
\item \texttt{unlock} \textit{name}\\
  Unlock the item with the given name (i.e., remove it from the set of
  locked items).  If any other client was waiting to lock this item,
  this should wake up one of them up so they can lock it.

\item \texttt{list}\\
  List all the items in the set of locked items (in any order).

\item \texttt{quit}\\
  This is a request for the server to terminate the connection with this
  client.  This behavior has already been implemented for you.

\end{itemize}

\subsubsection*{Client/Server Interaction}

  Client/server communication is all done in text.  We're using
  fdopen() to create a FILE pointer from the socket file descriptor.
  This lets us send and receive messages the same way we would
  write and read files using the C standard I/O library.  Of course,
  we could just read and write directly to the socket file descriptor,
  but using the C I/O functions should make sending and receiving text
  a little bit easier.

  The partial server implementation just repeatedly prompts the client
  for commands and terminates the connection when the client enters
  ``quit''.  You will extend the server to handle the commands
  described above.  There's a sample execution below to help demonstrate
  how the server is expected to respond to these commands.

\subsubsection*{Multi-Threading and Synchronization}

  Right now, the server uses just the main thread to accept new client
  connections and to communicate with the client.  It can only
  interact with one client at a time.  You're going to fix this by
  making it a multi-threaded server.  Each time a client connects, you
  will create a new thread to handle interaction with that client,
  using the pthread argument-passing mechanism to give the new thread
  the socket associated with that client's connection.

  With each client having its own thread on the server, there could be
  race conditions when threads try to access any state stored in the
  server (e.g., the set of items that are currently locked).  You'll
  need to add synchronization support to your server to prevent
  potential race conditions.  You can use POSIX semaphores or POSIX
  monitors for this (I suggest monitors).  That way, if two clients
  enter a command at the same time, the server will make sure their
  threads can't access shared server state at the same time.

  The commands described above require that the server makes a client
  wait if they try to lock an item that's already locked.  So,
  in your implementation of the lock command, you'll check to see if
  the requested item is currently in the set of locked items.  If it
  is, the server will force the thread associated with that client wait
  until the given item is unlocked by some other thread (maybe not the
  same one that locked it).  Whenever an item is unlocked, you server
  will remove it from the set of locked items and wake a thread that
  was waiting to lock that item (if such a thread exists).

\subsubsection*{Detaching Threads}

  Previously, we've always had the main thread join with the threads
  it created.  Here, we don't need to do that.  The main thread can
  just forget about a thread once it has created it.  Each new thread can
  communicate with its client and then terminate when it's done.

  For this to work properly, after creating a thread, the server needs
  to detach it using \texttt{pthread\_detach()}.  This tells pthreads
  to free memory for the given thread as soon as it terminates, rather
  than keeping it around for the benefit of another thread that's
  expected to join with it.

\subsubsection*{Buffer Overflow and Set Capacity}

  In its current state, the server is vulnerable to buffer overflow
  where it reads commands from the client.  You'll need to fix this
  vulnerability in the existing code and make sure you don't have
  potential buffer overflows in any new code you add.  Your server
  only needs to be able to handle the commands listed above.  For
  inputs that are too long, you can do whatever you want, print an
  error message, ignore the command, terminate the client connection,
  whatever.  No matter what a client sends it, it shouldn't permit a
  buffer overflow, since that could let an attacker get unintended
  behavior out of the server.

  An item name will consist of at most 24 characters, with no
  whitespace in the name.  There's no bound on the set of locked
  items; it could contain any number of different item names, and you
  don't know what any of these item names are going to be, \textit{a
    priori}.  You can store the set of locked items as a linked list
  or a resizable array, so there's no limit to the number of items
  that can be locked.

\subsubsection*{Port Numbers}

  Since we may be developing on multi-user systems, we need to make
  sure we all use different port numbers.  That way, a client written
  by one student won't accidentally try to connect to a server being
  written by another student.  To help prevent this, I've assigned
  each student their own port number.  Visit the following URL to find
  out what port number your server is supposed to use.

\url{people.engr.ncsu.edu/dbsturgi/class/info/246/}
  
\subsubsection*{Sad Truths about Sockets on EOS Linux Hosts}

  Since we're using TCP/IP for communication, we can finally run our
  client and server programs on different hosts.  You should be able
  to run your server on one host and then start nc on any other
  internet-connected system in the world, giving it the server's
  hostname and your port number on the command line.  However, if you
  try this on a typical university Linux machine, you'll probably be
  disappointed.  These systems have a software firewall that blocks
  most IP traffic.  As a result, if you want to develop on a
  university system, you will still need to run the client and server
  on the same host.  If you have your own Linux machine, you should be
  able to run your server on it and connect from anywhere (depending
  on how your network is set up).

  If you'd like to use a machine in the Virtual Computing Lab (VCL),
  you should be able to run commands as the administrator, so you can
  disable the software firewall.  I had success with a reservation for a
  CentOS 7 Base (64 bit VM) system.  After logging in, uploading and building my
  server, I ran the following command to permit TCP connections via my
  port.  Then, if I ran a server on the virtual machine, I was
  able to connect to it, even from off campus.

\begin{tabbing}
sudo /sbin/iptables -I INPUT 1 -p tcp \verb+--+dport \textit{my-port-number} -j ACCEPT
\end{tabbing}

  Also, if your server crashes, you may temporarily be unable
  to bind to your assigned port number, with an error message ``Can't
  bind socket.''  Just wait a minute or two and you should be able to
  run the server again.

\subsubsection*{Implementation}

  You will complete your server by extending the file,
  \texttt{lockServer.c} from the starter.  You'll need to compile as
  follows:

\begin{verbatim}
gcc -Wall -g -std=gnu99 -o lockServer lockServer.c -lpthread
\end{verbatim}

\subsubsection*{Sample Execution}

  You should be able to run your server with any number of concurrent
  clients, each handled by its own thread in the server.  Below, we're
  looking at three terminal sessions, with the server running in the
  left-hand column and clients running in the other two columns (be
  sure to use our own port number, instead of 28123).  I've
  interleaved the input/output to illustrate the order of interaction
  with each client, and I've added some comments to explain what's
  going on.  If you try this out on your own server, don't enter the
  comments, since the server doesn't know what to do with them.

\begin{verbatim}
$ ./lockServer

                  # Run one client and lock a
                  # couple of items.
                  $ nc localhost 28123
                  cmd> lock apple
                  cmd> lock banana

                                                 # Run another client
                                                 $ nc localhost 28123
                                                 # See what items are locked
                                                 cmd> list
                                                 apple
                                                 banana

                                                 # Lock another item.
                                                 cmd> lock cherry

                                                 # Client will block if it tries
                                                 # to lock a resource that's
                                                 # already locked.
                                                 cmd> lock banana

                  # Now, three items are locked
                  cmd> list        
                  apple
                  banana
                  cherry

                  # Unlocking this resource
                  # Will let the other client
                  # continue
                  cmd> unlock banana

                                                 cmd> lock durian

                  # The server doesn't keep up with
                  # who locked each resource.  You can
                  # unlock resources you locked, or
                  # resources locked by another client.
                  cmd> unlock apple
                  cmd> unlock cherry

                  # This will block this client.
                  cmd> lock durian

                                                 # Until we unlock 
                                                 cmd> unlock durian
                                                 cmd> unlock banana
                  cmd> quit

                                                 # Exit (leaving one resource
                                                 # still locked).
                                                 cmd> quit
\end{verbatim}

\subsubsection*{Submitting your Work}

When you're done, submit your \texttt{lockServer.c} source file using Wolfware Classic.

\item (40 pts) For this problem, we're going to write a distributed
  program that works with permissions information represented using a
  global table (an inefficient representation that you'd never really
  use, in practice).  This will give
  us a chance to try out UDP and to think about a simple model for how
  permissions could be stored.

\subsection*{Global Table Representation}

  At startup, the server will read a list of users from an input file
  called ``users.txt''.  This file contains one username per line, where
  the username is a sequence of non-whitespace characters.

  The server will read a list of objects from a file called
  ``objects.txt''.  This file also gives one object per line, where an
  object name is a sequence of non-whitespace characters.

  For the access rights, the server will read a list of triples from a
  file called ``table.txt''.  Each line of this file gives the number
  of a user from the user list (counting from zero), followed by the
  number of an object from the object list (counting from zero).  This
  is followed by a string that says an operation the given user is
  permitted to perform on the given object.  The name of an operation
  is a sequence of non-whitespace characters.  For example, the last
  line of the example ``table.txt'' is the following.  It says that
  Marge (user number 5) can delete the motorcycle (object number 8).

\begin{verbatim}
5 8 delete
\end{verbatim}

  You can assume the input files are all in the correct format.  We
  may test you program on different input files, but we won't test
  your solution on invalid files.  In the ``table.txt'' file, the same
  user/object pair may occur more than once, if a user can perform
  multiple operations on an object.

\subsection*{Partial Implementation}

  I'm giving you a mostly complete client, but you get to write the
  server yourself.  You can use our in-class example, UDPClient.java
  and UDPServer.java, to remind yourself of some of the details of how
  to communicate over UDP.  You'll find the partial implementation on
  the course homepage, PermissionsClient.java.

\subsection*{Query Commands}

  On the command line, the client expects a hostname where the server
  is running, the name of a user, the name of an object, and a desired
  operation.  It puts the last three parameters into a message, sends
  it to the server and waits for a response.  If the given user is
  permitted to perform the given operation on the given object, the
  server should respond and the client should print ``Access
  granted''.  If not, it should print ``Access denied''.  If the
  client provides a user name or an object name that aren't on the
  list, the server will just print ``Access denied''.  If the server
  doesn't respond within 2 seconds, the client should print ``No
  response''.

  Each time it's run, the client will encode a request in a UDP
  datagram and send it to the server's port on the given host.  The
  client will then wait for a response datagram, report the server's
  response to the user and then terminate.  If the user wants to run
  more commands, they can just run the client again.

\subsection*{Server Operation}

  After parsing the three input lists, the server will open a
  DatagramSocket and repeatedly process requests it receives from
  clients, sending back responses for correctly formed requests.  Your
  server won't need to be multi-threaded.

  You will have to kill the server manually (e.g., with ctrl-C) to get
  it to terminate.

\subsection*{Message Representation}

  The client concatenates the last three command-line arguments (space
  separated) into a string, converts the string to a byte array and
  sends it in a datagram.  It expects the server's response in the
  same format, a string encoded as a byte array.  You can use this
  format for messages if you want, or if you'd like to encode messages
  differently that's OK.  You'll just need to change the client to
  have it use your preferred message format.

\subsection*{Handling Bad Requests}

  If a malicious client tries to send the server a poorly-formed
  request, the server should print a message to standard output and
  just ignore the message (without even sending a response).  The
  Java exception mechanism gives us a good way to handle this.  If anything
  goes wrong while processing a request, it will throw an
  exception.  The server can catch this exception, print an error
  message of your choice and then go on immediately to wait for a
  datagram from the next client.

\subsection*{Lost Messages}

  UDP doesn't guarantee delivery of messages.  Either the
  client's request or the server's response may be lost.  Your client
  will deal with this by waiting at most 2 seconds for a response from
  the server.  If it doesn't get a response, it will just terminate
  and print ``No response''  The DatagramSocket class has a method,
  setSoTimeout(), that lets you specify how long you'd like to wait to
  receive a message.  You'll need to read about this method, figure
  out how to use it and add it to your client.

  You can test this behavior by killing your server, then running the
  client.  After waiting for two seconds, the client should give up
  and print``No response''.

\subsubsection*{Port Numbers}

  As in the C client/server program, you'll need to use your assigned
  port numbers for this program.  Here, the client and server will
  each need to use a different port
  numbers (so we have the option of running them on the same host).
  Just use your first assigned port number for your server and use the
  second one as your client's port number.  You can check the
  following URL to see what port numbers you've been assigned.

\url{people.engr.ncsu.edu/dbsturgi/class/info/246/}

\subsubsection*{Sad Truths about Sockets Remain}

  As with the C client/server, the software firewall on the EOS
  machines will prevent us from running our client and server on
  different hosts.  In developing your programs, you can just run
  client and server on the same host, giving ``localhost'' as the
  server's hostname.  If you have your own Linux system, you can try
  out your client/server on different hosts, or you can use a VCL
  machine and enable UDP traffic for the ports you're using.

  \subsection*{Sample Execution}

  Once your programs are working, you should be able to run them as
  follows, using the three data files provided with this assignment.
  Below, I've shown these commands as you would run them in two
  terminal windows on the same host, with the server execution on the
  left and several executions of the client on the right.  I've included
  some shell comments to say what's going on.

\begin{verbatim}
# start the server
$ java PermissionsServer

                   # The first access right in the table.
                   $ java PermissionsClient localhost Jerry inkwell append
                   Access granted

                   # The last one right in the table.
                   $ java PermissionsClient localhost Marge motorcycle delete
                   Access granted

                   # A user and object that aren't even on the lists.
                   $ java PermissionsClient localhost Susan truck park
                   Access denied

                   # A user and an object that are on our lists, but a right that's
                   # not grated in the table.
                   $ java PermissionsClient localhost Evie dumpster read
                   Access denied

                   # same user and object, but this time a right that's granted.
                   $ java PermissionsClient localhost Evie dumpster truncate
                   Access granted

# Kill the server
^C

                   # After a delay, the client gives up.
                   $ java PermissionsClient localhost Bill apple eat
                   No response
\end{verbatim}

\subsection*{Submitting your Work}

When you're done, submit an electronic copy of your
\texttt{PermissionsClient.java} and \texttt{PermissionsServer.java}
code using WolfWare classic.

\end{enumerate}

\end{document}

